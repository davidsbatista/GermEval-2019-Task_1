%
% File konvens2019.tex
%
% Contact: konferenz-konvens2019@lists.fau.de
%%
%% Based on the style files for KONVENS 2018, which were, in turn,
%% Based on the style files for KONVENS 2016, which were, in turn,
%% Based on the style files for GSCL-2015, which were, in turn,
%% Based on the style files for ACL-2014, which were, in turn,
%% Based on the style files for ACL-2013, which were, in turn,
%% Based on the style files for ACL-2012, which were, in turn,
%% based on the style files for ACL-2011, which were, in turn,
%% based on the style files for ACL-2010, which were, in turn,
%% based on the style files for ACL-IJCNLP-2009, which were, in turn,
%% based on the style files for EACL-2009 and IJCNLP-2008...

\documentclass[11pt]{article}
\usepackage{konvens2019}
\usepackage{mathptmx}
\usepackage[scaled=.90]{helvet}
\usepackage{courier}

\usepackage{url}
\usepackage{latexsym}

\usepackage{comment}

%\setlength\titlebox{5cm}

% You can expand the titlebox if you need extra space
% to show all the authors. Please do not make the titlebox
% smaller than 5cm (the original size); we will check this
% in the camera-ready version and ask you to change it back.


\title{COMTRAVO-DS team at GermEval 2019 shared task on hierarchical classification of blurbs}

\author{David S. Batista \\
  Comtravo Gmbh \\
  {\tt david.batista@comtravo.com} \\\And
  Second Author \\
  Comtravo Gmbh \\
  {\tt email@domain} \\}



\begin{document}
\maketitle
\begin{abstract}

\end{abstract}

\section{Introduction}

\section{Task}

The GermEval 2010 Shared Task on hierarchical classification of blurbs challenges involved the
classification of books into genres given a book's blurb i.e., a short textual description of the
book. The competition contained two sub-tasks:

\begin{itemize}

\item Sub-Task A: classify German books into one or multiple most general writing genres. Therfore,
it can be considered a multi-label classification task. In total, there are 8 classes that can be
assigned to a book.

\item Sub-Task B: targets hierarchical multi-label classification into multiple writing genres. In
addition to the very general writing genres, additional genres of different specificity can
be assigned to a book.

\end{itemize}


\subsection{Dataset}

The dataset made available for this task contains 3 levels and it's considered a hierarchically
multi-label since any example can be can assigned to more than one class at any given level of
the hierarchy. The hierarchy contains a totall of 343 classes distinct classes, and in total there
are 16 627 samples. Table~\ref{quantitivy-analysis-train} as a detailed statistical description of
the dataset.

% hierarchy-tree
% imbalence of classes ?
% taxonomy or tree ?
% Whether the taxonomy is organized into a tree or a DAG influences the degree of difficulty of
% the underlying hierarchical classification problem

\begin{table}
\begin{center}
\begin{tabular}{|l|r|}
\hline\centering\textbf{Training dataset}  &                    \\
\hline
Avg. lenght of blurb (tokens)              &  96.78             \\
Std. deviation $\sigma$ (tokens)           &  39.63             \\
Avg. lenght of blurb (sentences)           &  6.55              \\
Std. deviation $\sigma$ (sentences)        &  2.76              \\
\hline
Total number of genres                     &  343               \\
Possible genres per level (1;2;3)          &  8; 93; 242        \\
Avg. genres per per blurb                  &  3.1               \\
Std. deviation $\sigma$                    &  1.36              \\
Avg. genres per blurb at level (1;2;3)     &  1.06; 1.34; 0.69  \\
Std. deviation $\sigma$                    &  0.27; 0.76; 0.79  \\
\hline
Avg. blurb per co-occorrence               &                    \\
Co-occorrence std. deviation               &                    \\
Leaf nodes at each level (1;2;3)           &                    \\
\hline
\end{tabular}
\end{center}
\caption{\label{quantitivy-analysis-train}Quantitative analysis of the training dataset.}
\end{table}


\begin{table}
\begin{center}
\begin{tabular}{|l|r|}
\hline\centering\textbf{Test dataset}  &         \\
\hline
Avg. lenght of blurb (tokens)              &  96.91             \\
Std. deviation $\sigma$ (tokens)           &  39.83             \\
Avg. lenght of blurb (sentences)           &  6.55              \\
Std. deviation $\sigma$ (sentences)        &  2.62              \\
\hline
\end{tabular}
\end{center}
\caption{\label{quantitivy-analysis-test}Quantitative analysis of the test dataset.}
\end{table}


% TODO: grafico com histogram de freqs. para tokens/sentence/genre ?


\textbf{TODO:}


\section{Hierarchical Classification}

\textbf{TODO:}

\subsection{Local Classifier}

The local classifier strategy is one way to approach the hierarchical document classification task
and it was first proposed, to the best of our knowledge, in the seminal work of Koller and
Sahami~\shortcite{Koller:1997:HCD:645526.657130}, it is also sometimes refered to as top-down
approach in the literature.

There are different approaches, based on the idea of a local classifier, depending on how they use
the local information and devise a strategy to build several classifiers~\cite{Silla:2011:SHC:1937796.1937884}:

\begin{itemize}
\item{a local classifier per node}
\item{a local classifier per parent node}
\item{a local classifier per level}
\end{itemize}


% Local Classifier Per Node Approach
The \textit{local classifier per node} approach consists in training one binary classifier for each
node in the hierarchy tree, where node every possible label in the hierarchy tree. There are several
diferent strategies on how to select the positive and negative classes (see~\cite{})


% Local Classifier Per Parent Node Approach
In \textit{a per parent node approach}, for each parent node in the hierarchy tree, a multi-class
classifier is trained to classify the probability of a given sample beloging to each of the parent's
child nodes.

%During the testing phase, this approach is often coupled with the top-down class prediction approach,
%but this coupling is not necessarily a must, as new class prediction approaches for this type of
%local approach could be developed.

%Consider the top-down class-prediction approach and the same class tree example of Figure 5,
%and suppose that the first level classifier assigns the example to the class 2.

%The second level classifier, which was only trained with the children of the class node 2, in this
%case 2.1 and 2.2, will then make its class assignment (and so on, if deeper-level classifiers were available),
%therefore avoiding the problem of making inconsistent predictions and respecting the natural constrains of class membership.


% Local Classifier Per Level Approach
The \textit{local classifier per level} approach consists of training one multi-class classifier
for each level of the hierarchy tree. When a new test sample is presented we get the output of the
classifiers from each level and use this information as the final classification.

This approach however is prone to class-membership inconsistency, since different classifiers are
trained for each level of the hierarchy. For instance, it's possible to have has outputs for the
first-level classes 2 and 3, ant then classes 1.1 and 1.2 at the second-level, and class 3.1.1
at the third-level, whichi generates an inconsistent classification. This approach should
be complemented by a post-processing method that tries to correct the prediction inconsistency.


% Summary

\begin{comment}
In essence, in this top-down approach, for each new example in the test set, the system first
predicts its first-level (most generic) class, then it uses that predicted class to narrow
the choices of classes to be predicted at the second level (the only valid candidate
second-level classes are the children of the class predicted at the first level), and so on,
recursively, until the most specific prediction is made.

This local approach has some disadvantages, a disadvantage of the top-down class-prediction
approach (which is shared by all the three types of local classifiers discussed next) is that an
error at a certain class level is going to be propagated downwards the hierarchy, unless some
procedure for avoiding this problem is used
\end{comment}



\subsection{Global Classifier: label co-ocorruence}



\section{System}

\subsection{Preprocessing}

% consider only alphanumeric tokens
% stop-words ?

\subsection{Approaches}


\section{Experiments and Results}

\subsection{Local Classifier Per Parent Node Approach}




\section*{Acknowledgments}

%The acknowledgments should go immediately before the references.
%Do not number the acknowledgments section. Do not include this section
%when submitting your paper for review.

\bibliographystyle{konvens2019}
\bibliography{references.bib}

\end{document}
