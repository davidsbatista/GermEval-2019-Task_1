%
% File acl2019.tex
%
%% Based on the style files for ACL 2018, NAACL 2018/19, which were
%% Based on the style files for ACL-2015, with some improvements
%%  taken from the NAACL-2016 style
%% Based on the style files for ACL-2014, which were, in turn,
%% based on ACL-2013, ACL-2012, ACL-2011, ACL-2010, ACL-IJCNLP-2009,
%% EACL-2009, IJCNLP-2008...
%% Based on the style files for EACL 2006 by 
%%e.agirre@ehu.es or Sergi.Balari@uab.es
%% and that of ACL 08 by Joakim Nivre and Noah Smith

\documentclass[11pt,a4paper]{article}
\usepackage[hyperref]{acl2019}
\usepackage{times}
\usepackage{latexsym}
\usepackage{comment}

\usepackage{url}

\aclfinalcopy % Uncomment this line for the final submission
%\def\aclpaperid{***} %  Enter the acl Paper ID here

%\setlength\titlebox{5cm}
% You can expand the titlebox if you need extra space
% to show all the authors. Please do not make the titlebox
% smaller than 5cm (the original size); we will check this
% in the camera-ready version and ask you to change it back.

\newcommand\BibTeX{B\textsc{ib}\TeX}

\title{COMTRAVO\_DS team at GermEval 2019 shared task on hierarchical classification of blurbs}

\author{First Author \\
  Affiliation / Address line 1 \\
  Affiliation / Address line 2 \\
  Affiliation / Address line 3 \\
  \texttt{email@domain} \\\And
  Second Author \\
  Affiliation / Address line 1 \\
  Affiliation / Address line 2 \\
  Affiliation / Address line 3 \\
  \texttt{email@domain} \\}

\begin{document}

\maketitle

\begin{abstract}

\end{abstract}

\section{Introduction}

\section{Task}

The GermEval 2010 Shared Task on hierarchical classification of blurbs challenges involved the
classification of books into genres given a book's blurb i.e., a short textual description of the
book.

The competition contained two tasks:

\begin{itemize}

\item Task 1: classify German books into one or multiple most general writing genres. Therfore,
it can be considered a multi-label classification task. In total, there are 8 classes that can be
assigned to a book.

\item Task 2: targets hierarchical multi-label classification into multiple writing genres. In
addition to the very general writing genres, additional genres of different specificity can
be assigned to a book. In total, there are 343 different classes that are hierarchically
structured on up to 4 levels.

\end{itemize}


\subsection{Dataset}

There are XX XXX samples in the dataset provided by the shared task.

\section{System}


\subsection{Preprocessing}


\section*{Acknowledgments}

\section*{References}

\bibliography{acl2019}
\bibliographystyle{acl_natbib}


\end{document}
